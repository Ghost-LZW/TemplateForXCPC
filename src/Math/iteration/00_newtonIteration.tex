\begin{enumerate}

\item describe

具体的任务是,对于在 $[a,b]$ 上连续且单调的函数 $f(x)$,求方程 $f(x)=0$ 的近似解。

初始时我们从给定的 $f(x)$ 和一个近似解 $x_0$ 开始。($x_0$ 的值可任意取)

假设我们目前的近似解是 $x_i$,我们画出与 $f(x)$ 切于点 $(x_i,f(x_i))$ 的直线 $l$,将 $l$ 与 $x$ 轴的交点横坐标记为 $x_{i+1}$,那么这就是一个更优的近似解。重复这个迭代的过程。
根据导数的几何意义,可以得到如下关系:

$$
 f^\prime(x_i) = \frac{f(x_i)}{x_{i} - x_{i+1}}
$$

整理后得到如下递推式:

$$
 x_{i+1} = x_i - \frac{f(x_i)}{f^\prime(x_i)}
$$

直观地说,如果 $f(x)$ 比较平滑,那么随着迭代次数的增加,$x_i$ 会越来越逼近方程的解。

牛顿迭代法的收敛率是平方级别的,这意味着每次迭代后近似解的精确数位会翻倍。

\item 求解平方根

我们尝试用牛顿迭代法求解平方根。设 $f(x)=x^2-n$,这个方程的近似解就是 $\sqrt{n}$ 的近似值。于是我们得到

$$
x_{i+1}=x_i-\frac{x_i^2-n}{2x_i}=\frac{x_i+\frac{n}{x_i}}{2}
$$

在实现的时候注意设置合适的精度。代码如下

\begin{lstlisting}
double sqrt_newton(double n) {
  const double eps = 1E-15;
  double x = 1;
  while (true) {
    double nx = (x + n / x) / 2;
    if (abs(x - nx) < eps) break;
    x = nx;
  }
  return x;
}
\end{lstlisting}

\item 求解整数平方根

尽管我们可以调用 `sqrt()` 函数来获取平方根的值,但这里还是讲一下牛顿迭代法的变种算法,用于求不等式 $x^2\le n$ 的最大整数解。我们仍然考虑一个类似于牛顿迭代的过程,但需要在边界条件上稍作修改。如果 $x$ 在迭代的过程中上一次迭代值得近似解变小,而这一次迭代使得近似解变大,那么我们就不进行这次迭代,退出循环。

\begin{lstlisting}
int isqrt_newton(int n) {
  int x = 1;
  bool decreased = false;
  for (;;) {
    int nx = (x + n / x) >> 1;
    if (x == nx || (nx > x && decreased)) break;
    decreased = nx < x;
    x = nx;
  }
  return x;
}
\end{lstlisting}

\item 高精度平方根

最后考虑高精度的牛顿迭代法。迭代的方法是不变的,但这次我们需要关注初始时近似解的设置,即 $x_0$ 的值。由于需要应用高精度的数一般都非常大,因此不同的初始值对于算法效率的影响也很大。一个自然的想法就是考虑 $x_0=2^{\left\lfloor\frac{1}{2}\log_2n\right\rfloor}$,这样既可以快速计算出 $x_0$,又可以较为接近平方根的近似解。

给出 Java 代码的实现:

\begin{lstlisting}
public static BigInteger isqrtNewton(BigInteger n) {
  BigInteger a = BigInteger.ONE.shiftLeft(n.bitLength() / 2);
  boolean p_dec = false;
  for (;;) {
    BigInteger b = n.divide(a).add(a).shiftRight(1);
    if (a.compareTo(b) == 0 || a.compareTo(b) < 0 && p_dec)
      break;
    p_dec = a.compareTo(b) > 0;
    a = b;
  }
  return a;
}
\end{lstlisting}

实践效果:在 $n=10^{1000}$ 的时候该算法的运行时间是 60 ms,如果我们不优化 $x_0$ 的值,直接从 $x_0=1$ 开始迭代,那么运行时间将增加到 120 ms。

\end{enumerate}